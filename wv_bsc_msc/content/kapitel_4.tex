% Kapitel
\chapter{Infrastruktur}
\label{chap:infrastruktur}

\epigraph{\glqq The secret of all victory lies in the organization of the non-obvious.\grqq\bigskip}%\textsc
{{Marcus Aurelius}\\ (121--180)}

\noindent
Die Erstellung einer Arbeit sollte in einer bereitgestellten Infrastruktur erfolgen, die insbesondere bei der Planung und Verwaltung einer Arbeit unterstützt.
\indent

\section{GitLab-Server}
\label{gitlab}

Unterschiedliche Dienste, die im Kontext einer Arbeit von Nutzen sind, werden über einen GitLab-Server bereitgestellt. Jeder Kandidat erhält einen
persönlichen Zugang und ein eigenes Repository. In diesem Repository werden Dokumente und eigene Inhalte der Arbeit zentral verwaltet und somit dem
Betreuer zur Kontrolle übergeben. Der GitLab-Server stellt hierzu in erster Linie ein Repository bereit. Das Repository bzw. eine Versionsverwaltung
im Allgemeinen hilft vor allem bei der Verwaltung von textbasierten Dateien, so z.\,B. Quellcode oder Dokumente in \LaTeX .\\

Zur Planung einer Arbeit und Kontrolle des Fortschritts erfolgt das Projektmanagement digital innerhalb von GitLab. Hierzu werden Milestones und Issues
angelegt und während des Projektes gepflegt bzw. Fortschritte kontrolliert. Eine möglichst präzise Projektplanung hilft bei der Vermeidung von etwaigen
zeitlichen Engpässen im Laufe der Erstellung einer Arbeit.

\section{Versionsverwaltung mit Git}

Die Verwaltung der Arbeit, die mittels \LaTeX\ verfasst wird, und aller zugehörigen Dateien bzw. Dokumente kann auf einfache und sehr transparente
Weise mittels einer Versionsverwaltung erfolgen. Als Versionsverwaltung wird Git\footnote{\url{https://git-scm.com/}, aufgerufen am 16.06.2017} eingesetzt.
Git steht für alle gängigen Betriebssysteme bereit.\\

Jegliche Änderungen und Ergänzungen werden von Git erkannt und aufgezeichnet. Erfolgte Änderungen sollten mittels sog. \glqq Commits\grqq\ eingepflegt und beschrieben werden.
Die Versionsverwaltung erfolgt in erster Linie auf dem lokalen System. Erfolgte Änderungen bzw. Fortschritte sollten -- nicht nur als Backup -- regelmäßig über den
bereitgestellten GitLab-Server dem Betreuer zur Verfügung gestellt werden.

Die Arbeit mit Git kann sowohl auf der Kommandozeile als auch in Applikation mit UI erfolgen. Die Applikation
\glqq SourceTree\grqq\footnote{\url{https://www.sourcetreeapp.com/}, aufgerufen am 16.06.2017} ermöglicht beispielsweise
die komfortable Verwaltung von Git-Repositories.

Grundsätzliche Tipps zum Umgang mit Git liefern die offizielle Dokumentation\footnote{\url{https://git-scm.com/doc}, aufgerufen am 16.06.2017}
und das \glqq Git Cheat Sheet\grqq\footnote{\url{https://www.git-tower.com/blog/git-cheat-sheet/}, aufgerufen am 16.06.2017}.

Gerade im Zusammenhang mit \LaTeX entstehen viele temporäre Dateien, die nicht in der Versionsverwaltung landen sollten.
Dazu sollte eine \texttt{gitignore} Konfiguration\footnote{\url{https://www.gitignore.io/}, aufgerufen am 04.07.2017} erstellt werden.