% Kapitel
\chapter{Einleitung}

\epigraph{\glqq The user's going to pick dancing pigs over security every time.\grqq\bigskip}
{\textsc{Bruce Schneier}\\ ($\ast$1963)}

\noindent
Das Verfassen eines Praxissemesterberichts \dots
\indent

\section{Hinweise}

Bitte denken Sie daran, dass Sie die eidesstattliche Erklärung vor Abgabe unterschreiben.

\section{Inhalt}

Der Praxissemesterbericht MUSS -- neben dem Hauptteil (1 Seite Bericht pro Kalenderwoche) --
nachfolgende Inhalte berücksichtigen:

\begin{itemize}
  \item Titelseite
  \item Eidesstattliche Erklärung
  \item Zusammenfassung und Abstract (Englisch)
  \item Inhaltsverzeichnis, Abbildungsverzeichnis, Tabellenverzeichnis, Abkürzungs\-ver\-zeichnis und Literaturverzeichnis
  \item Einleitung
  \item Fazit bestehend aus einer reflektierten Zusammenfassung
\end{itemize}

\section{Organisatorisches}

\begin{itemize}
 \item Es gilt die jeweils aktuelle Pr\"ufungsordnung (\S 15 in BMI PO vom 04.08.2010 bzw. \S 15 in MMI PO vom 16.06.2011).
 Lesen Sie aufmerksam die für Sie geltende Prüfungsordnung und richten Sie sich nach den dort
 definierten Vorgaben (es sei denn Sie haben mit dem Prüfer eine Abweichung abgesprochen).
 \item Abzugeben gebunden als Ausdruck (beidseitig bedruckt) und elektronisch als PDF
\end{itemize}

\section{Bewertungskriterien}

Anhand des Praxissemesterberichts und einem Fachgespräch, welches einen Kurzvortrag im Umfang von 15 Minuten
duch den Kandidaten / die Kandidatin beinhaltet, entscheidet der Prüfer, ob das Praxissemester akzeptiert
wird oder nicht. Eine Benotung findet nicht statt.