% Kapitel
\chapter{Tools}

\epigraph{\glqq Man is still the most extraordinary computer of all.\grqq\bigskip}%\textsc
{{John F. Kennedy}\\ (1917--1963)}

\noindent
Nachfolgende Hinweise und Empfehlungen zum Einsatz von Tools vereinfachen den Umgang mit \LaTeX.
\indent

\section{\LaTeX}

Das Verfassen von Dokumenten mit \LaTeX\ kann durch unterschiedlichste Tools unterstützt werden. Da \LaTeX\ grundsätzlich textbasiert
arbeitet können jegliche Inhalte auch in einem einfachen Texteditor erstellt und angepasst werden.\\

Mittels unterschiedlichster Editoren kann die Erstellung und Pflege von Dokumenten mit \LaTeX\ vereinfacht werden. Unter Linux bietet
der Editor \glqq Kile\grqq\footnote{\url{http://kile.sourceforge.net/}, aufgerufen am 16.06.2017} eine Vielzahl nützlicher Funktionen.
Für Apple OS X und Microsoft Windows ist \glqq Texmaker\grqq\footnote{\url{http://www.xm1math.net/texmaker/}, aufgerufen am 16.06.2017}
empfehlenswert. Unbedingt empfohlen wird die Verwendung einer Rechtschreibprüfung, die typischerweise in den Editoren integriert sind.